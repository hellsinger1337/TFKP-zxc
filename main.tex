\documentclass[12pt,a4paper]{article}

% Подключаем пакеты
\usepackage[utf8]{inputenc} % Кодировка текста
\usepackage[T2A]{fontenc} % Поддержка кириллицы
\usepackage[russian,english]{babel} % Многоязычная поддержка
\usepackage{amsmath,amsfonts,amssymb} % Математические символы и шрифты
\usepackage{geometry} % Настройка полей страницы
\usepackage{polynom} %деление многочленов столбиком
\usepackage{amsmath}
\usepackage{amssymb}
\usepackage{tikz}

% Настройка полей страницы
\geometry{left=2cm}
\geometry{right=2cm}
\geometry{top=2cm}
\geometry{bottom=2cm}

\title{Рассчетка по теории функций комплексного переменного}
\author{Крылов Савелий Игоревич ИБ-21}
\date{\today}

\begin{document}

\maketitle
\newpage

\renewcommand{\contentsname}{Оглавление}
\tableofcontents
\newpage

% Введение
\section*{Введение}
Я делал задание на бумаге -- бумага потерялась, решил избавиться от проблемы(бумаги).

% Глава 1
\section{Комплексные числа и действия с ними}
\subsection*{1.10}
\[
\text{Re} \left( \frac{1 - i}{2 - i} \right) - \text{Im} \left( \frac{2 + i}{1 + 3i} \right)
\]
\subsubsection*{Решение}
\[
\text{Re} \left( \frac{1-i}{2-i} \right) - \text{Im} \left( \frac{2+i}{1+3i} \right) = \text{Re} \left( \frac{(1-i)(2+i)}{5} \right) - \text{Im} \left( \frac{(2+i)(1-3i)}{10} \right) =
\]

\[
= \text{Re} \left( \frac{2-i+1}{5} \right) - \text{Im} \left( \frac{2-2i+3)}{10} \right) = \frac{3}{5}+\frac{1}{5}i
\]
\subsection*{1.20}
\[
z^2 + \overline{z} = 2\operatorname{Re} \left( \frac{1 + i}{1 - i} \right)
\]
\subsubsection*{Решение}
\[
z^2 + \overline{z} = 2 \operatorname{Re} \left( \frac{1+i}{1-i} \right)
\]
\[
z = x + iy
\]
\[
x^2 + 2xyi + iy^2 - y^2 + x - yi = 2 \operatorname{Re} \left( \frac{1+2i-1}{2} \right)
\]
\[
x^2 - y^2 + x + i(2x-1)y = 0
\]

\[
\left\{
\begin{aligned}
x^2 - y^2 + x &= 0, \\
(2x - 1) y &= 0.
\end{aligned}
\right.
\]

Находим решения для \( y = 0 \):
\begin{align*}
x^2 + x &= 0 \\
(x + 1)x &= 0 \\
x &= 0 \quad \text{or} \quad x = -1.
\end{align*}

Для случая \( (2x-1) = 0 \) (считая \( y \neq 0 \)):
\begin{align*}
2x - 1 &= 0, \\
x &= \frac{1}{2}.
\end{align*}
Подставляем \( x = \frac{1}{2} \) в первое равенство:
\begin{align*}
\left(\frac{1}{2}\right)^2 - y^2 + \frac{1}{2} &= 0, \\
\frac{1}{4} - y^2 + \frac{1}{2} &= 0, \\
y^2 &= \frac{3}{4}, \\
y &= \pm \frac{\sqrt{3}}{2}.
\end{align*}

Решения:
\begin{itemize}
\item \((\frac{1}{2}, \frac{\sqrt{3}}{2})\),
\item \((\frac{1}{2}, -\frac{\sqrt{3}}{2})\),
\item \((0, 0)\),
\item \((-1, 0)\).
\end{itemize}
\subsection*{1.30}
\[
\sqrt{-3 - 4i}
\]
\subsubsection*{Решение}
\[
\sqrt{-3-4i}
\]
\[
w = -3 - 4i
\]
\[
|w| = \sqrt{(-3)^2 + (-4)^2} = 5
\]
\[
w = 5(\cos\phi + i\sin\phi)
\]
\[
\cos\phi = -\frac{3}{5}, \quad \sin\phi = -\frac{4}{5}
\]

\[
z=\sqrt{w} = \sqrt{5} \left( \cos \left( \frac{\phi + 2k\pi}{2} \right) + i\sin \left( \frac{\phi + 2k\pi}{2} \right) \right), \quad k = 0, 1
\]
\[
z_1 = \sqrt{5} \left( \cos \left( \frac{\phi}{2} \right) + i\sin \left( \frac{\phi}{2} \right) \right)
\]
\[
z_2=-z_1
\]
\[
2\cos^{2} \left( \frac{\phi}{2} \right) - 1 = \cos\phi = -\frac{3}{5}
\]
\[
2\cos^2 \left( \frac{\phi}{2} \right)  = \frac{2}{5}
\]
\[
\cos \left( \frac{\phi}{2} \right) = \sqrt{\frac{1}{5}}
\]
\[
\sin \left( \frac{\phi}{2} \right) = \sqrt{1 - \cos^2 \left( \frac{\phi}{2} \right)} = -\sqrt{\frac{4}{5}}
\]
\[
z_1 = \sqrt{5} \left( \frac{1}{\sqrt{5}} - i\frac{2}{\sqrt{5}} \right) = 1 - 2i
\]
\[
z_2 = -\sqrt{5} \left( \frac{1}{\sqrt{5}} - i\frac{2}{\sqrt{5}} \right) = -1 + 2i
\]
\subsection*{1.40}
\[
 z^2+(-1+6i)z-6i=0
\]
\subsubsection*{Решение}
\[
z = 1 + 0i \text{ -- решение(очевидно понимается подстановкой)}
\]
\begin{tabular}{r|l}

$z^2 + (1 + 6i)^2 - 6i$ & $z-1$ \\
\cline{2-2}
$z^2-z$~~~~~~~~~~~~~~~~~& $z+6i $ \\
\cline{1-1}
$6iz - 6i$ & \\
$6iz - 6i$ & \\
\cline{1-1}
0&\\
\end{tabular}
\[
(z-1)(z + 6i) = 0
\]
\[
z = 1 \text{ или } z = -6i
\]


\subsection*{1.50}

 Представить комплексное число $\sqrt{3}+i$ в тригонометрической и показательных формах

\subsubsection*{Решение}
\[
z = \sqrt{3} + i
\]
\[
|z| = \sqrt{\sqrt{3}^2 + 1^2} = 2
\]
\[
\text{Re} > 0, \quad \text{Im} > 0
\]
\[
\arg(z) = \phi = \arctan\left(\frac{1}{\sqrt{3}}\right) = \frac{\pi}{6}
\]
\[
 z = 2 \left(\cos \frac{\pi}{6} + i \sin \frac{\pi}{6}\right)
\]
\[
z = |z| \cdot e^{i\phi} = 2 \cdot e^{i\frac{\pi}{6}}
\]




\subsection*{1.60}

 Доказать неравенство, используя свойства модуля комплексного числа


\subsubsection*{Решение}
\[
|\frac{1-3z^2\overline{z}}{2+i}| < 12 \quad (\text{если } |z| \leq 2)
\]
\[
|1-3z^2\overline{z}| < 12\sqrt{5} \quad (\text{домножили на } |{2+i}|= \sqrt{5})
\]
\[
|1-3z^2\overline{z}| \leq |1| + |3z^2\overline{z}|=1+3|z|^3 \leq 1+3*8=25 < 12\sqrt{5}
\]

\[
25 < 12\sqrt{5}
\]
\[
6.25 < 14.5
\]
\[
625 < 144*5
\]
\[
625 < 720
\]




\subsection*{1.70}
Вычислить $(\frac{1-i}{\sqrt{3}-i})^{20}$ применяя тригонометрическую форму комплексного числа 


\subsubsection*{Решение}
\[
z = \frac{1-i}{\sqrt{3}-i}
\]
\[
|1-i| = \sqrt{1^2 + (-1)^2} = \sqrt{2}, \quad \arg(1-i) = -\frac{\pi}{4}
\]
\[
|\sqrt{3} - i| = \sqrt{(\sqrt{3})^2 + (-1)^2} = 2, \quad \arg(\sqrt{3} - i) = -\frac{\pi}{6}
\]
\[
\left| \frac{1-i}{\sqrt{3}-i} \right| = \frac{\sqrt{2}}{2}
\]
\[
\arg\left( \frac{1-i}{\sqrt{3}-i} \right) = -\frac{\pi}{4} - (-\frac{\pi}{6}) = -\frac{\pi}{12}
\]
\[
z = \frac{\sqrt{2}}{2} \left( \cos\left(-\frac{\pi}{12}\right) + i \sin\left(-\frac{\pi}{12}\right) \right)
\]
\[
z^{20} = \left(\frac{\sqrt{2}}{2}\right)^{20} \left( \cos\left(20 \cdot -\frac{\pi}{12}\right) + i \sin\left(20 \cdot -\frac{\pi}{12}\right) \right)
\]
\[
= \frac{1}{2^{10}} \left( \cos\left(-\frac{5\pi}{3}\right) + i \sin\left(-\frac{5\pi}{3}\right) \right)
\]
\[
= \frac{1}{1024} \left( \cos\left(\frac{\pi}{3}\right) + i \sin\left(\frac{\pi}{3}\right) \right)
\]
\[
= \frac{1}{1024} \left( \frac{1}{2} + i \frac{\sqrt{3}}{2} \right)=\frac{1+i\sqrt{3}}{2048}
\]



\subsection*{1.80}
Вычислить $\frac{-2\sqrt{3} - 2i}{1 - i}^{\frac{1}{7}}$ применяя тригонометрическую форму комплексного числа 


\subsubsection*{Решение}
\[
z = \frac{-2\sqrt{3} - 2i}{1 - i}
\]
\[
|1 - i| = \sqrt{1^2 + (-1)^2} = \sqrt{2}, \quad \arg(1-i) = -\frac{\pi}{4}
\]
\[
|-2\sqrt{3} - 2i| = \sqrt{(-2\sqrt{3})^2 + (-2)^2} = 4, \quad \arg(-2\sqrt{3} - 2i) = -\frac{2\pi}{3}
\]
\[
\left| \frac{-2\sqrt{3} - 2i}{1 - i} \right| = \frac{4}{\sqrt{2}} = 2\sqrt{2}
\]
\[
\arg\left( \frac{-2\sqrt{3} - 2i}{1 - i} \right) = -\frac{2\pi}{3} - (-\frac{\pi}{4}) = -\frac{5\pi}{12}
\]
\[
z = 2\sqrt{2} \left( \cos\left(-\frac{5\pi}{12}\right) + i \sin\left(-\frac{5\pi}{12}\right) \right)
\]
\[
z^{1/7} = \left(2\sqrt{2}\right)^{1/7} \left( \cos\left(\frac{-\frac{5\pi}{12} + 2k\pi}{7}\right) + i \sin\left(\frac{-\frac{5\pi}{12} + 2k\pi}{7}\right) \right), \quad k = 0, 1, \dots, 6
\]






%глава 2
\section{Изображение комплексных чисел}
\subsection*{2.10}
Изобразить на комлексной плоскости множество точек $z \in \mathbb{C}$, удовлетворяюзих системе соотножений $4 \leq |z-1|+|z+1| \leq 8$
\subsubsection*{Решение}
\begin{itemize}
\item Внутренний эллипс (минимальное расстояние 4) представляет собой множество точек, каждая из которых имеет суммарное расстояние до двух фокусов равное 4. Это минимальный эллипс.
\item Внешний эллипс (максимальное расстояние 8) аналогично описывает множество точек, суммарное расстояние до фокусов которых равно 8.
\end{itemize}

\begin{center}
\begin{tikzpicture}


\begin{scope}
\clip (0,0) ellipse (4 and 3.87298334621); 
\fill[gray!50] (0,0) ellipse (4 and 3.87298334621); 
\fill[white] (0,0) ellipse (2 and 1.73205080757); 
\end{scope}

\draw[gray!30, thin, step=0.5] (-5,-4) grid (5,4);
\draw[thick,->] (-5,0) -- (5,0) node[right] {$\text{Re}$};
\draw[thick,->] (0,-4) -- (0,4) node[above] {$\text{Im}$};

\draw[<->,red,thick] (-1,0) -- (1,0);
\draw[<->,blue,thick] (0,-1) --  (0,1);
\draw (0,0) ellipse (2 and 1.73205080757);
\draw (0,0) ellipse (4 and 3.87298334621);

\node at (-1,-0.3) {$-1$};
\node at (1,-0.3) {$1$};
\node at (0.3,-1) {$-i$};
\node at (0.3,1) {$i$};

\end{tikzpicture}
\end{center}


\subsection*{2.20}
Изобразить линию, заданную уравнением $2z\overline{z}+(2+i)z+(2-i)\overline{z}=2$
\subsubsection*{Решение}
Уравнение в комплексной форме:
\[
2z\overline{z} + (2+i)z + (2-i)\overline{z} = 2
\]
Преобразуется в:
\[
2(x^2 + y^2) + (2x + y) +(2y+x)i +(2x - y) - (2y+x)i = 2
\]
\[
x^2 + y^2 + 2x - y = 1
\]
Получаем уравнение окружности:
\[
(x+1)^2 + \left(y-\frac{1}{2}\right)^2 = \left(\frac{3}{2}\right)^2
\]
Центр окружности: \(-1+\frac{1}{2}i\), радиус: \(\frac{3}{2}\).

\begin{center}
\begin{tikzpicture}[scale=2]
    \draw[step=0.5,gray,very thin] (-3,-1) grid (1,2);
    \draw[thick,->] (-3,0) -- (1,0) node[right] {$\text{Re}$};
    \draw[thick,->] (0,-1) -- (0,2) node[above] {$\text{Im}$};

    % Окружность
    \draw[thick,blue] (-1,0.5) circle (1.5);
    \filldraw[black] (-1,0.5) circle (1pt) node[anchor=north east] {$-1+\frac{i}{2}$};

    % Радиус
    \draw[red, thick] (-1,0.5) -- node[above left] {$\frac{3}{2}$} (-2.5,0.5);
\end{tikzpicture}
\end{center}

%глава 3
\section{Функции $e^z,\sin{z},\cos{z},\cosh{z},\sinh{z},\sqrt[n]{z},Ln(z),\ln{z}$}
\subsection*{3.10}
Вычислить $\text{Im} \exp{2-5i}$
\subsubsection*{Решение}
\[
e^{2-5i} = e^{2}(\cos(-5) + i\sin(-5))
\]
\[
\text{Im } e^{2-5i} = e^2 \sin(-5)
\]

\subsection*{3.20}
Найти действительную,мнимую части и модуль функции $f(z)=\sin(z)$
\subsubsection*{Решение}
\[
\sin(z) =\sin(x+iy)= \sin(x)\cosh(y) + i\cos(x)\sinh(y)
\]
\[
\text{Re}(\sin(z)) = \sin(x) \cosh(y)
\]
\[
\text{Im}(\sin(z)) = \cos(x) \sinh(y)
\]
\[
|\sin(z)| = \sqrt{(\sin(x) \cosh(y))^2 + (\cos(x) \sinh(y))^2}=
\]
\[
= \sqrt{(\sin(x) \cosh(y))^2 + (1-\sin^2(x))( \sinh(y))^2}=\sqrt{\sinh^2{x}+\sin^2{x}}
\]

\subsection*{3.30}
Вычислить значения логарифмов $Ln 10,\ln(10)$
\subsubsection*{Решение}
\[
Ln(10) = \ln{10}+i\arg(10)+i*2k\pi, k \in \mathbf{Z}
\]
\[
\arg(10)=0
\]
\[
Ln(10) = \ln{10}+i*2k\pi, k \in \mathbf{Z}
\]
\[
\ln(10+0i)=\ln{10}
\]

\subsection*{3.40}
Найти все значения выражения $(-4)^{-4}$
\subsubsection*{Решение}
\[
\arg{-4}=\pi
\]
\[
\sqrt[4]{-4}=\sqrt[4]{4}*(\cos(\frac{\pi+2k\pi}{4})+i\sin(\frac{\pi+2k\pi}{4})),k=0,1,2,3
\]
\[
z_0=\sqrt{2}*(\cos{\frac{\pi}{4}}+i\sin{\frac{\pi}{4}})=\sqrt{2}*(\frac{\sqrt{2}}{2}+i\frac{\sqrt{2}}{2})=i+1
\]
\[
z_0=i+1,z_1=i-1,z_2=-i-1,z_3=-i+1
\]


%глава 4
\section{Дробно-линейная функция}
\subsection*{4.10}
Найти дробно-линейнойное отображение $w=f(z)$, которое переводит точки $z_1=-1,z_2=0,z_3=1$ в точки $w_1=1,w_2=i,w_3=-1$ соответственно
\subsubsection*{Решение}
\[
\frac{a(-1) + b}{c(-1) + d} = 1, \quad \frac{b}{d} = i, \quad \frac{a + b}{c + d} = -1
\]
\[
\frac{-a + b}{-c + d} = 1 \Rightarrow -a + b = -c + d
\]
\[
\frac{b}{d} = i \Rightarrow b = id
\]
\[
\text{Подставим } b = id 
\]
\[
\frac{a + id}{c + d} = -1 \Rightarrow a + id = -c - d
\]
\[
-a + id = -c + d \Rightarrow a + c = i(d - b)
\]
\[
\text{Из уравнения } a + id = -c - d
\]
\[
a = -c - d - id
\]
\[
\text{Пусть } c = 1, d = 1
\]
\[
b = i, \quad a = -2 - i
\]
\[
f(z) = \frac{(-2-i)z + i}{z + 1}
\]

\subsection*{4.20}
Найти дробно-линейнойную функцию $w=f(z)$, отображающую $G$ - круг $|z-1| \leq 1$ на
$H$ - полуплоскость $\text{Im}(z) \leq 2$
\subsubsection*{Решение}
\[
f(z) = \frac{az + b}{cz + d}
\]
\[
z_1=0,z_2=2,z_3=1+i \text{ в точки } w_1=2i-2,w_2=2i+2,w_3=2i
\]
\[
\frac{b}{d} = 2i - 2, \quad \frac{a(1+i) + b}{c(1+i) + d} = 2i, \quad \frac{2a + b}{2c + d} = 2 + 2i
\]

\[
b = (2i - 2)d, \quad a(1+i) + b = 2i(c(1+i) + d), \quad 2a + b = 2(2c + d) + 2i(2c + d)
\]
\[
\text{Пусть } d = 1 \text{, тогда } b = 2i - 2
\]
\[
a(1+i) + (2i - 2) = 2i(c(1+i) + 1), \quad 2a + (2i - 2) = 2 + 2i + 4c + 2i
\]

\[
a(1+i) - 2 = 2ic - 2c + 2i, \quad 2a = 4 + 4c
\]

\[
 c = -1, a = 2i
\]

\[
f(z) = \frac{(2i)z + (2i - 2)}{-z + 1}
\]


%глава 5
\section{Дифференцирование функций комплексного переменного}
\subsection*{5.10}
Проверить выполнение условий Коши-Римана для функции $f(z)=ze^z$
\subsubsection*{Решение}

\[
f(z) = z e^z = (x+iy) e^{x+iy}
\]

Действительная и мнимая части:
\[
u(x, y) = x e^x \cos y - y e^x \sin y, \quad v(x, y) = x e^x \sin y + y e^x \cos y
\]

Частные производные:
\[
\frac{\partial u}{\partial x} = e^x \cos y + x e^x \cos y - y e^x \sin y
\]
\[
\frac{\partial u}{\partial y} = -x e^x \sin y - y e^x \cos y - e^x \sin y
\]
\[
\frac{\partial v}{\partial x} = e^x \sin y + x e^x \sin y + y e^x \cos y
\]
\[
\frac{\partial v}{\partial y} = x e^x \cos y - y e^x \sin y + e^x \cos y
\]
\[
\frac{\partial u}{\partial x} = \frac{\partial v}{\partial y}, \quad \frac{\partial u}{\partial y} = -\frac{\partial v}{\partial x}
\]

Условия Коши-Римана выполнены.

\subsection*{5.20}
Определить точки дифференцируемости и область аналитичности функции $f(z)=f(x+iy)=2xy+iy^2$. Найти производную в точках существования.
\subsubsection*{Решение}
\[
  u(x, y) = 2xy
 \]
 \[
v(x, y) = y^2
 \]

Условия Коши-Римана:
\[
\frac{\partial u}{\partial x} = 2y, \quad \frac{\partial u}{\partial y} = 2x
\]
\[
\frac{\partial v}{\partial x} = 0, \quad \frac{\partial v}{\partial y} = 2y
\]
\[
\]
\[
\frac{\partial u}{\partial x} = \frac{\partial v}{\partial y} \quad \text{(всегда) и} \quad \frac{\partial u}{\partial y} = -\frac{\partial v}{\partial x} \quad (\text{выполняется только если } x = 0)
\]

Функция дифференцируема и аналитична вдоль мнимой оси:
\[
f'(iy) = \frac{\partial u}{\partial x}+i\frac{\partial v}{\partial x}=2y+i*0=2y
\]
\subsection*{5.30}
Найти аналитическую функцию $f(z)=u(x,y)+iv(x,y)$ по ее указанным свойствам 
$v(x,y)=4\sinh{x}\sin{y}+2xy, f(0)=3$
\subsubsection*{Решение}
Для функции \( v(x, y) = 4\sinh x \sin y + 2xy \), используя условия Коши-Римана:
\[
\frac{\partial v}{\partial x} = 4\cosh x \sin y + 2y, \quad \frac{\partial v}{\partial y} = 4\sinh x \cos y + 2x
\]
\[
\frac{\partial u}{\partial x} = \frac{\partial v}{\partial y}, \quad \frac{\partial u}{\partial y} = -\frac{\partial v}{\partial x}
\]
При интегрировании по \( x \) для \( \frac{\partial u}{\partial x} \):
\[
u(x, y) = \int (4 \sinh x \cos y + 2x) \, dx = 4 \cos y \cosh x + x^2 + \lambda(y)
\]

Дифференцируем по \( y \):
\[
(4 \cos y \cosh x + x^2 + \lambda(y))'_{y}=-4*\sin{y}\cosh{x}+\lambda(y)'_{y}
\]
Вспоминаем условия Коши-Римана:
\[
-4*\sin{y}\cosh{x}+\lambda(y)'_{y}=-\frac{\partial v}{\partial x}=-4*\sin{y}\cosh{x}-2y
\]
\[
\lambda(y)'_{y}=-2y
\]
\[
\lambda(y)=-y^2+C
\]
Получаем, что:
\[
u(x, y) = 4\cosh x \sin y + x^2 - y^2 + C
\]
Из начального условия \( f(0) = 3 \):
\[
u(0, 0) = 3 -v(0,0)
\]
\[
v(0, 0) = 4\sinh 0 \sin 0 + 2*0*0=0 
\]
\[
3=u(0, 0) = 4\cosh 0 \sin 0 + 0^2 - 0^2 + C
\]
\[
3=C
\]
Таким образом, аналитическая функция:
\[
f(z) = (4\cosh x \sin y + x^2 - y^2 + 3) + i(4\sinh x \sin y + 2xy)
\]
\end{document}