\documentclass[12pt,a4paper]{article}

% Подключаем пакеты
\usepackage[utf8]{inputenc} % Кодировка текста
\usepackage[T2A]{fontenc} % Поддержка кириллицы
\usepackage[russian,english]{babel} % Многоязычная поддержка
\usepackage{amsmath,amsfonts,amssymb} % Математические символы и шрифты
\usepackage{geometry} % Настройка полей страницы
\usepackage{polynom} %деление многочленов столбиком
\usepackage{amsmath}
\usepackage{amssymb}
\usepackage{tikz}

% Настройка полей страницы
\geometry{left=2cm}
\geometry{right=2cm}
\geometry{top=2cm}
\geometry{bottom=2cm}

\title{Рассчетка по теории функций комплексного переменного}
\author{Крылов Савелий Игоревич ИБ-21}
\date{\today}

\begin{document}

\maketitle
\newpage

\renewcommand{\contentsname}{Содержание}
\tableofcontents
\newpage

% Введение
\section*{Введение}
Я делал задание на бумаге -- бумага потерялась, решил избавиться от проблемы(бумаги).

% Глава 1
\section{Комплексные числа и действия с ними}
\subsection*{1.10}
\[
\text{Re} \left( \frac{1 - i}{2 - i} \right) - \text{Im} \left( \frac{2 + i}{1 + 3i} \right)
\]
\subsubsection*{Решение}
\[
\text{Re} \left( \frac{1-i}{2-i} \right) - \text{Im} \left( \frac{2+i}{1+3i} \right) = \text{Re} \left( \frac{(1-i)(2+i)}{5} \right) - \text{Im} \left( \frac{(2+i)(1-3i)}{10} \right) =
\]

\[
= \text{Re} \left( \frac{2-i+1}{5} \right) - \text{Im} \left( \frac{2-2i+3)}{10} \right) = \frac{3}{5}+\frac{1}{5}i
\]
\subsection*{1.20}
\[
z^2 + \overline{z} = 2\operatorname{Re} \left( \frac{1 + i}{1 - i} \right)
\]
\subsubsection*{Решение}
\[
z^2 + \overline{z} = 2 \operatorname{Re} \left( \frac{1+i}{1-i} \right)
\]
\[
z = x + iy
\]
\[
x^2 + 2xyi + iy^2 - y^2 + x - yi = 2 \operatorname{Re} \left( \frac{1+2i-1}{2} \right)
\]
\[
x^2 - y^2 + x + i(2x-1)y = 0
\]

\[
\left\{
\begin{aligned}
x^2 - y^2 + x &= 0, \\
(2x - 1) y &= 0.
\end{aligned}
\right.
\]

Находим решения для \( y = 0 \):
\begin{align*}
x^2 + x &= 0 \\
(x + 1)x &= 0 \\
x &= 0 \quad \text{or} \quad x = -1.
\end{align*}

Для случая \( (2x-1) = 0 \) (считая \( y \neq 0 \)):
\begin{align*}
2x - 1 &= 0, \\
x &= \frac{1}{2}.
\end{align*}
Подставляем \( x = \frac{1}{2} \) в первое равенство:
\begin{align*}
\left(\frac{1}{2}\right)^2 - y^2 + \frac{1}{2} &= 0, \\
\frac{1}{4} - y^2 + \frac{1}{2} &= 0, \\
y^2 &= \frac{3}{4}, \\
y &= \pm \frac{\sqrt{3}}{2}.
\end{align*}

Решения:
\begin{itemize}
\item \((\frac{1}{2}, \frac{\sqrt{3}}{2})\),
\item \((\frac{1}{2}, -\frac{\sqrt{3}}{2})\),
\item \((0, 0)\),
\item \((-1, 0)\).
\end{itemize}
\subsection*{1.30}
\[
\sqrt{-3 - 4i}
\]
\subsubsection*{Решение}
\[
\sqrt{-3-4i}
\]
\[
w = -3 - 4i
\]
\[
|w| = \sqrt{(-3)^2 + (-4)^2} = 5
\]
\[
w = 5(\cos\phi + i\sin\phi)
\]
\[
\cos\phi = -\frac{3}{5}, \quad \sin\phi = -\frac{4}{5}
\]

\[
z=\sqrt{w} = \sqrt{5} \left( \cos \left( \frac{\phi + 2k\pi}{2} \right) + i\sin \left( \frac{\phi + 2k\pi}{2} \right) \right), \quad k = 0, 1
\]
\[
z_1 = \sqrt{5} \left( \cos \left( \frac{\phi}{2} \right) + i\sin \left( \frac{\phi}{2} \right) \right)
\]
\[
z_2=-z_1
\]
\[
2\cos^{2} \left( \frac{\phi}{2} \right) - 1 = \cos\phi = -\frac{3}{5}
\]
\[
2\cos^2 \left( \frac{\phi}{2} \right)  = \frac{2}{5}
\]
\[
\cos \left( \frac{\phi}{2} \right) = \sqrt{\frac{1}{5}}
\]
\[
\sin \left( \frac{\phi}{2} \right) = \sqrt{1 - \cos^2 \left( \frac{\phi}{2} \right)} = -\sqrt{\frac{4}{5}}
\]
\[
z_1 = \sqrt{5} \left( \frac{1}{\sqrt{5}} - i\frac{2}{\sqrt{5}} \right) = 1 - 2i
\]
\[
z_2 = -\sqrt{5} \left( \frac{1}{\sqrt{5}} - i\frac{2}{\sqrt{5}} \right) = -1 + 2i
\]
\subsection*{1.40}
\[
 z^2+(-1+6i)z-6i=0
\]
\subsubsection*{Решение}
\[
z = 1 + 0i \text{ -- решение(очевидно понимается подстановкой)}
\]
\begin{tabular}{r|l}

$z^2 + (1 + 6i)^2 - 6i$ & $z-1$ \\
\cline{2-2}
$z^2-z$~~~~~~~~~~~~~~~~~& $z+6i $ \\
\cline{1-1}
$6iz - 6i$ & \\
$6iz - 6i$ & \\
\cline{1-1}
0&\\
\end{tabular}
\[
(z-1)(z + 6i) = 0
\]
\[
z = 1 \text{ или } z = -6i
\]


\subsection*{1.50}

 Представить комплексное число $\sqrt{3}+i$ в тригонометрической и показательных формах

\subsubsection*{Решение}
\[
z = \sqrt{3} + i
\]
\[
|z| = \sqrt{\sqrt{3}^2 + 1^2} = 2
\]
\[
\text{Re} > 0, \quad \text{Im} > 0
\]
\[
\arg(z) = \phi = \arctan\left(\frac{1}{\sqrt{3}}\right) = \frac{\pi}{6}
\]
\[
 z = 2 \left(\cos \frac{\pi}{6} + i \sin \frac{\pi}{6}\right)
\]
\[
z = |z| \cdot e^{i\phi} = 2 \cdot e^{i\frac{\pi}{6}}
\]




\subsection*{1.60}

 Доказать неравенство, используя свойства модуля комплексного числа


\subsubsection*{Решение}
\[
|\frac{1-3z^2\overline{z}}{2+i}| < 12 \quad (\text{если } |z| \leq 2)
\]
\[
|1-3z^2\overline{z}| < 12\sqrt{5} \quad (\text{домножили на } |{2+i}|= \sqrt{5})
\]
\[
|1-3z^2\overline{z}| \leq |1| + |3z^2\overline{z}|=1+3|z|^3 \leq 1+3*8=25 < 12\sqrt{5}
\]

\[
25 < 12\sqrt{5}
\]
\[
6.25 < 14.5
\]
\[
625 < 144*5
\]
\[
625 < 720
\]




\subsection*{1.70}
Вычислить $(\frac{1-i}{\sqrt{3}-i})^{20}$ применяя тригонометрическую форму комплексного числа 


\subsubsection*{Решение}
\[
z = \frac{1-i}{\sqrt{3}-i}
\]
\[
|1-i| = \sqrt{1^2 + (-1)^2} = \sqrt{2}, \quad \arg(1-i) = -\frac{\pi}{4}
\]
\[
|\sqrt{3} - i| = \sqrt{(\sqrt{3})^2 + (-1)^2} = 2, \quad \arg(\sqrt{3} - i) = -\frac{\pi}{6}
\]
\[
\left| \frac{1-i}{\sqrt{3}-i} \right| = \frac{\sqrt{2}}{2}
\]
\[
\arg\left( \frac{1-i}{\sqrt{3}-i} \right) = -\frac{\pi}{4} - (-\frac{\pi}{6}) = -\frac{\pi}{12}
\]
\[
z = \frac{\sqrt{2}}{2} \left( \cos\left(-\frac{\pi}{12}\right) + i \sin\left(-\frac{\pi}{12}\right) \right)
\]
\[
z^{20} = \left(\frac{\sqrt{2}}{2}\right)^{20} \left( \cos\left(20 \cdot -\frac{\pi}{12}\right) + i \sin\left(20 \cdot -\frac{\pi}{12}\right) \right)
\]
\[
= \frac{1}{2^{10}} \left( \cos\left(-\frac{5\pi}{3}\right) + i \sin\left(-\frac{5\pi}{3}\right) \right)
\]
\[
= \frac{1}{1024} \left( \cos\left(\frac{\pi}{3}\right) + i \sin\left(\frac{\pi}{3}\right) \right)
\]
\[
= \frac{1}{1024} \left( \frac{1}{2} + i \frac{\sqrt{3}}{2} \right)=\frac{1+i\sqrt{3}}{2048}
\]



\subsection*{1.80}
Вычислить $\frac{-2\sqrt{3} - 2i}{1 - i}^{\frac{1}{7}}$ применяя тригонометрическую форму комплексного числа 


\subsubsection*{Решение}
\[
z = \frac{-2\sqrt{3} - 2i}{1 - i}
\]
\[
|1 - i| = \sqrt{1^2 + (-1)^2} = \sqrt{2}, \quad \arg(1-i) = -\frac{\pi}{4}
\]
\[
|-2\sqrt{3} - 2i| = \sqrt{(-2\sqrt{3})^2 + (-2)^2} = 4, \quad \arg(-2\sqrt{3} - 2i) = -\frac{2\pi}{3}
\]
\[
\left| \frac{-2\sqrt{3} - 2i}{1 - i} \right| = \frac{4}{\sqrt{2}} = 2\sqrt{2}
\]
\[
\arg\left( \frac{-2\sqrt{3} - 2i}{1 - i} \right) = -\frac{2\pi}{3} - (-\frac{\pi}{4}) = -\frac{5\pi}{12}
\]
\[
z = 2\sqrt{2} \left( \cos\left(-\frac{5\pi}{12}\right) + i \sin\left(-\frac{5\pi}{12}\right) \right)
\]
\[
z^{1/7} = \left(2\sqrt{2}\right)^{1/7} \left( \cos\left(\frac{-\frac{5\pi}{12} + 2k\pi}{7}\right) + i \sin\left(\frac{-\frac{5\pi}{12} + 2k\pi}{7}\right) \right), \quad k = 0, 1, \dots, 6
\]






%глава 2
\section{Изображение комплексных чисел}
\subsection*{2.10}
Изобразить на комлексной плоскости множество точек $z \in \mathbb{C}$, удовлетворяюзих системе соотножений $4 \leq |z-1|+|z+1| \leq 8$
\subsubsection*{Решение}
\begin{itemize}
\item Внутренний эллипс (минимальное расстояние 4) представляет собой множество точек, каждая из которых имеет суммарное расстояние до двух фокусов равное 4. Это минимальный эллипс.
\item Внешний эллипс (максимальное расстояние 8) аналогично описывает множество точек, суммарное расстояние до фокусов которых равно 8.
\end{itemize}

\begin{center}
\begin{tikzpicture}


\begin{scope}
\clip (0,0) ellipse (4 and 3.87298334621); 
\fill[gray!50] (0,0) ellipse (4 and 3.87298334621); 
\fill[white] (0,0) ellipse (2 and 1.73205080757); 
\end{scope}

\draw[gray!30, thin, step=0.5] (-5,-4) grid (5,4);
\draw[thick,->] (-5,0) -- (5,0) node[right] {$\text{Re}$};
\draw[thick,->] (0,-4) -- (0,4) node[above] {$\text{Im}$};

\draw[<->,red,thick] (-1,0) -- (1,0);
\draw[<->,blue,thick] (0,-1) --  (0,1);
\draw (0,0) ellipse (2 and 1.73205080757);
\draw (0,0) ellipse (4 and 3.87298334621);

\node at (-1,-0.3) {$-1$};
\node at (1,-0.3) {$1$};
\node at (0.3,-1) {$-i$};
\node at (0.3,1) {$i$};

\end{tikzpicture}
\end{center}


\subsection*{2.20}
Изобразить линию, заданную уравнением $2z\overline{z}+(2+i)z+(2-i)\overline{z}=2$
\subsubsection*{Решение}
Уравнение в комплексной форме:
\[
2z\overline{z} + (2+i)z + (2-i)\overline{z} = 2
\]
Преобразуется в:
\[
2(x^2 + y^2) + (2x + y) +(2y+x)i +(2x - y) - (2y+x)i = 2
\]
\[
x^2 + y^2 + 2x - y = 1
\]
Получаем уравнение окружности:
\[
(x+1)^2 + \left(y-\frac{1}{2}\right)^2 = \left(\frac{3}{2}\right)^2
\]
Центр окружности: \(-1+\frac{1}{2}i\), радиус: \(\frac{3}{2}\).

\begin{center}
\begin{tikzpicture}[scale=2]
    \draw[step=0.5,gray,very thin] (-3,-1) grid (1,2);
    \draw[thick,->] (-3,0) -- (1,0) node[right] {$\text{Re}$};
    \draw[thick,->] (0,-1) -- (0,2) node[above] {$\text{Im}$};

    % Окружность
    \draw[thick,blue] (-1,0.5) circle (1.5);
    \filldraw[black] (-1,0.5) circle (1pt) node[anchor=north east] {$-1+\frac{i}{2}$};

    % Радиус
    \draw[red, thick] (-1,0.5) -- node[above left] {$\frac{3}{2}$} (-2.5,0.5);
\end{tikzpicture}
\end{center}

%глава 3
\section{Функции $e^z,\sin{z},\cos{z},\cosh{z},\sinh{z},\sqrt[n]{z},Ln(z),\ln{z}$}
\subsection*{3.10}
Вычислить $\text{Im} \exp{2-5i}$
\subsubsection*{Решение}
\[
e^{2-5i} = e^{2}(\cos(-5) + i\sin(-5))
\]
\[
\text{Im } e^{2-5i} = e^2 \sin(-5)
\]

\subsection*{3.20}
Найти действительную,мнимую части и модуль функции $f(z)=\sin(z)$
\subsubsection*{Решение}
\[
\sin(z) =\sin(x+iy)= \sin(x)\cosh(y) + i\cos(x)\sinh(y)
\]
\[
\text{Re}(\sin(z)) = \sin(x) \cosh(y)
\]
\[
\text{Im}(\sin(z)) = \cos(x) \sinh(y)
\]
\[
|\sin(z)| = \sqrt{(\sin(x) \cosh(y))^2 + (\cos(x) \sinh(y))^2}=
\]
\[
= \sqrt{(\sin(x) \cosh(y))^2 + (1-\sin^2(x))( \sinh(y))^2}=\sqrt{\sinh^2{x}+\sin^2{x}}
\]

\subsection*{3.30}
Вычислить значения логарифмов $Ln 10,\ln(10)$
\subsubsection*{Решение}
\[
Ln(10) = \ln{10}+i\arg(10)+i*2k\pi, k \in \mathbf{Z}
\]
\[
\arg(10)=0
\]
\[
Ln(10) = \ln{10}+i*2k\pi, k \in \mathbf{Z}
\]
\[
\ln(10+0i)=\ln{10}
\]

\subsection*{3.40}
Найти все значения выражения $(-4)^{-4}$
\subsubsection*{Решение}
\[
\arg{-4}=\pi
\]
\[
\sqrt[4]{-4}=\sqrt[4]{4}*(\cos(\frac{\pi+2k\pi}{4})+i\sin(\frac{\pi+2k\pi}{4})),k=0,1,2,3
\]
\[
z_0=\sqrt{2}*(\cos{\frac{\pi}{4}}+i\sin{\frac{\pi}{4}})=\sqrt{2}*(\frac{\sqrt{2}}{2}+i\frac{\sqrt{2}}{2})=i+1
\]
\[
z_0=i+1,z_1=i-1,z_2=-i-1,z_3=-i+1
\]


%глава 4
\section{Дробно-линейная функция}
\subsection*{4.10}
Найти дробно-линейнойное отображение $w=f(z)$, которое переводит точки $z_1=-1,z_2=0,z_3=1$ в точки $w_1=1,w_2=i,w_3=-1$ соответственно
\subsubsection*{Решение}
\[
\frac{a(-1) + b}{c(-1) + d} = 1, \quad \frac{b}{d} = i, \quad \frac{a + b}{c + d} = -1
\]
\[
\frac{-a + b}{-c + d} = 1 \Rightarrow -a + b = -c + d
\]
\[
\frac{b}{d} = i \Rightarrow b = id
\]
\[
\text{Подставим } b = id 
\]
\[
\frac{a + id}{c + d} = -1 \Rightarrow a + id = -c - d
\]
\[
-a + id = -c + d \Rightarrow a + c = i(d - b)
\]
\[
\text{Из уравнения } a + id = -c - d
\]
\[
a = -c - d - id
\]
\[
\text{Пусть } c = 1, d = 1
\]
\[
b = i, \quad a = -2 - i
\]
\[
f(z) = \frac{(-2-i)z + i}{z + 1}
\]

\subsection*{4.20}
Найти дробно-линейнойную функцию $w=f(z)$, отображающую $G$ - круг $|z-1| \leq 1$ на
$H$ - полуплоскость $\text{Im}(z) \leq 2$
\subsubsection*{Решение}
\[
f(z) = \frac{az + b}{cz + d}
\]
\[
z_1=0,z_2=2,z_3=1+i \text{ в точки } w_1=2i-2,w_2=2i+2,w_3=2i
\]
\[
\frac{b}{d} = 2i - 2, \quad \frac{a(1+i) + b}{c(1+i) + d} = 2i, \quad \frac{2a + b}{2c + d} = 2 + 2i
\]

\[
b = (2i - 2)d, \quad a(1+i) + b = 2i(c(1+i) + d), \quad 2a + b = 2(2c + d) + 2i(2c + d)
\]
\[
\text{Пусть } d = 1 \text{, тогда } b = 2i - 2
\]
\[
a(1+i) + (2i - 2) = 2i(c(1+i) + 1), \quad 2a + (2i - 2) = 2 + 2i + 4c + 2i
\]

\[
a(1+i) - 2 = 2ic - 2c + 2i, \quad 2a = 4 + 4c
\]

\[
 c = -1, a = 2i
\]

\[
f(z) = \frac{(2i)z + (2i - 2)}{-z + 1}
\]


%глава 5
\section{Дифференцирование функций комплексного переменного}
\subsection*{5.10}
Проверить выполнение условий Коши-Римана для функции $f(z)=ze^z$
\subsubsection*{Решение}

\[
f(z) = z e^z = (x+iy) e^{x+iy}
\]

Действительная и мнимая части:
\[
u(x, y) = x e^x \cos y - y e^x \sin y, \quad v(x, y) = x e^x \sin y + y e^x \cos y
\]

Частные производные:
\[
\frac{\partial u}{\partial x} = e^x \cos y + x e^x \cos y - y e^x \sin y
\]
\[
\frac{\partial u}{\partial y} = -x e^x \sin y - y e^x \cos y - e^x \sin y
\]
\[
\frac{\partial v}{\partial x} = e^x \sin y + x e^x \sin y + y e^x \cos y
\]
\[
\frac{\partial v}{\partial y} = x e^x \cos y - y e^x \sin y + e^x \cos y
\]
\[
\frac{\partial u}{\partial x} = \frac{\partial v}{\partial y}, \quad \frac{\partial u}{\partial y} = -\frac{\partial v}{\partial x}
\]

Условия Коши-Римана выполнены.

\subsection*{5.20}
Определить точки дифференцируемости и область аналитичности функции $f(z)=f(x+iy)=2xy+iy^2$. Найти производную в точках существования.
\subsubsection*{Решение}
\[
  u(x, y) = 2xy
 \]
 \[
v(x, y) = y^2
 \]

Условия Коши-Римана:
\[
\frac{\partial u}{\partial x} = 2y, \quad \frac{\partial u}{\partial y} = 2x
\]
\[
\frac{\partial v}{\partial x} = 0, \quad \frac{\partial v}{\partial y} = 2y
\]
\[
\]
\[
\frac{\partial u}{\partial x} = \frac{\partial v}{\partial y} \quad \text{(всегда) и} \quad \frac{\partial u}{\partial y} = -\frac{\partial v}{\partial x} \quad (\text{выполняется только если } x = 0)
\]

Функция дифференцируема и аналитична вдоль мнимой оси:
\[
f'(iy) = \frac{\partial u}{\partial x}+i\frac{\partial v}{\partial x}=2y+i*0=2y
\]
\subsection*{5.30}
Найти аналитическую функцию $f(z)=u(x,y)+iv(x,y)$ по ее указанным свойствам 
$v(x,y)=4\sinh{x}\sin{y}+2xy, f(0)=3$
\subsubsection*{Решение}
Для функции \( v(x, y) = 4\sinh x \sin y + 2xy \), используя условия Коши-Римана:
\[
\frac{\partial v}{\partial x} = 4\cosh x \sin y + 2y, \quad \frac{\partial v}{\partial y} = 4\sinh x \cos y + 2x
\]
\[
\frac{\partial u}{\partial x} = \frac{\partial v}{\partial y}, \quad \frac{\partial u}{\partial y} = -\frac{\partial v}{\partial x}
\]
При интегрировании по \( x \) для \( \frac{\partial u}{\partial x} \):
\[
u(x, y) = \int (4 \sinh x \cos y + 2x) \, dx = 4 \cos y \cosh x + x^2 + \lambda(y)
\]

Дифференцируем по \( y \):
\[
(4 \cos y \cosh x + x^2 + \lambda(y))'_{y}=-4*\sin{y}\cosh{x}+\lambda(y)'_{y}
\]
Вспоминаем условия Коши-Римана:
\[
-4*\sin{y}\cosh{x}+\lambda(y)'_{y}=-\frac{\partial v}{\partial x}=-4*\sin{y}\cosh{x}-2y
\]
\[
\lambda(y)'_{y}=-2y
\]
\[
\lambda(y)=-y^2+C
\]
Получаем, что:
\[
u(x, y) = 4\cosh x \sin y + x^2 - y^2 + C
\]
Из начального условия \( f(0) = 3 \):
\[
u(0, 0) = 3 -v(0,0)
\]
\[
v(0, 0) = 4\sinh 0 \sin 0 + 2*0*0=0 
\]
\[
3=u(0, 0) = 4\cosh 0 \sin 0 + 0^2 - 0^2 + C
\]
\[
3=C
\]
Таким образом, аналитическая функция:
\[
f(z) = (4\cosh x \sin y + x^2 - y^2 + 3) + i(4\sinh x \sin y + 2xy)
\]
%глава 5
\section{Интегрирование функций комплексного переменного}
\subsection*{6.10}
Вычислить интеграл
\[
\int_\Gamma f(z) \, dz,
\]
$\Gamma$ --- отрезок с концами в точках \( z_1 \) и \( z_2 \), проходимый от \( z_1 \) к \( z_2 \).
\[
f(z)=z\overline{z}+4z+1, z1=-1-i,z2=1+i
\]
\subsubsection*{Решение}
Параметризуем отрезок \(\Gamma\) как
\[
z(t) = z_1 + t(z_2 - z_1) = (-1 - i) + t(2 + 2i), \quad t \in [0, 1]
\]
Тогда
\[
z(t) = -1 - i + 2t + 2ti = (2t - 1) + i(2t - 1)
\]
и
\[
\frac{dz}{dt} = 2 + 2i
\]

Теперь вычислим \(\overline{z(t)}\):
\[
\overline{z(t)} = \overline{(2t - 1) + i(2t - 1)} = (2t - 1) - i(2t - 1)
\]

Тогда
\[
f(z(t)) = z(t)\overline{z(t)} + 4z(t) + 1
\]

Вычислим \(z(t)\overline{z(t)}\):
\[
z(t)\overline{z(t)} = [(2t - 1) + i(2t - 1)][(2t - 1) - i(2t - 1)] = (2t - 1)^2 + (2t - 1)^2 = 2(2t - 1)^2
\]

Тогда,
\[
f(z(t)) = 2(2t - 1)^2 + 4[(2t - 1) + i(2t - 1)] + 1
\]
\[
f(z(t)) = 2(2t - 1)^2 + 4(2t - 1) + 4i(2t - 1) + 1
\]

Интеграл преобразуется в
\[
\int_\Gamma f(z) \, dz = \int_0^1 f(z(t)) \frac{dz}{dt} \, dt
\]
\[
= \int_0^1 [2(2t - 1)^2 + 4(2t - 1) + 4i(2t - 1) + 1](2 + 2i) \, dt
\]
\[
= \int_0^1 [2(2t - 1)^2(2 + 2i) + 4(2t - 1)(2 + 2i) + 4i(2t - 1)(2 + 2i) + (2 + 2i)] \, dt
\]
\[
= \int_0^1 \left[4(2t - 1)^2 + 4i(2t - 1)^2 + 8(2t - 1) + 8i(2t - 1) + 8i(2t - 1) - 8(2t - 1) + 2 + 2i \right] \, dt
\]
\[
= \int_0^1 \left[ 4(2t - 1)^2 + 4i(2t - 1)^2 + 8i(2t - 1) + 2 + 2i \right] \, dt
\]
Вычислим каждый член отдельно:
\[
\int_0^1 4(2t - 1)^2 \, dt + \int_0^1 4i(2t - 1)^2 \, dt + \int_0^1 8i(2t - 1) \, dt + \int_0^1 2 \, dt + \int_0^1 2i \, dt
\]
Вычислим каждый интеграл:
\[
\int_0^1 4(2t - 1)^2 \, dt = 4 \int_0^1 (4t^2 - 4t + 1) \, dt = 4 \left( \left[ \frac{4t^3}{3} - 2t^2 + t \right]_0^1 \right) = 4 \left( \frac{4}{3} - 2 + 1 \right) = 4 \left( \frac{1}{3} \right) = \frac{4}{3}
\]

\[
\int_0^1 4i(2t - 1)^2 \, dt = 4i \int_0^1 (4t^2 - 4t + 1) \, dt = 4i \left( \frac{1}{3} \right) = \frac{4i}{3}
\]

\[
\int_0^1 8i(2t - 1) \, dt = 8i \left[ \frac{(2t - 1)^2}{2} \right]_0^1 = 8i \left( \frac{1}{2} - \frac{1}{2} \right) = 0
\]

\[
\int_0^1 2 \, dt = 2 \left[ t \right]_0^1 = 2
\]

\[
\int_0^1 2i \, dt = 2i \left[ t \right]_0^1 = 2i
\]

Сложим всё вместе:
\[
\int_\Gamma f(z) \, dz = \frac{4}{3} + \frac{4i}{3} + 2 + 2i = \left( \frac{4}{3} + 2 \right) + \left( \frac{4i}{3} + 2i \right) = \frac{10}{3} + \frac{10i}{3} = \frac{10}{3}(1 + i)
\]

Итак, ответ:
\[
\int_\Gamma f(z) \, dz = \frac{10}{3}(1 + i)
\]
\subsection*{6.20}
Вычислить интеграл
\[
\int_{0}^{i} \cosh z \, dz
\]
\subsubsection*{Решение}

\[
\frac{d\sinh z}{dz}  = \cosh z
\]

Таким образом,
\[
\int \cosh z \, dz = \sinh z + C
\]

Теперь вычислим определенный интеграл:
\[
\int_{0}^{i} \cosh z \, dz = \left. \sinh z \right|_{0}^{i} = \sinh(i) - \sinh(0)
\]

\(\sinh(0) = 0\),
\[
\int_{0}^{i} \cosh z \, dz = \sinh(i)
\]

\[
\sinh z = \frac{e^z - e^{-z}}{2}
\]
Для \(z = i\):
\[
\sinh(i) = \frac{e^i - e^{-i}}{2}
\]

\[
e^i = \cos(1) + i\sin(1)
\]
\[
e^{-i} = \cos(1) - i\sin(1)
\]

\[
\sinh(i) = \frac{(\cos(1) + i\sin(1)) - (\cos(1) - i\sin(1))}{2} = \frac{2i\sin(1)}{2} = i\sin(1)
\]
\[
\int_{0}^{i} \cosh z \, dz = i\sin(1)
\]
\subsection*{6.30}
Вычислить интеграл с помощью интегральной формулы Коши

\(\int_{\Gamma} \frac{\cos z}{(z + 1)(z + i)} \, dz,\)

a) \(\Gamma\) --- окружность \(|z - 1 - i| = 8\); \quad b) \(\Gamma\) --- окружность \(|z - 1 - i| = 4\).

\subsubsection*{Решение}
\[
\int_{\Gamma} \frac{\cos z}{(z + 1)(z + i)} \, dz
\]
и вычислим его с помощью интегральной формулы Коши.

a) \(\Gamma\) --- окружность \(|z - 1 - i| = 8\)

Простые нули \( \frac{\cos z}{(z + 1)(z + i)} \) находятся в точках \( z = -1 \) и \( z = -i \). Оба простых нуля находятся внутри окружности \(|z - 1 - i| = 8\).

 Интеграл вокруг \( z = -1 \)

Используем интегральную формулу Коши для функции \( f(z) = \frac{\cos z}{z + i} \) с \( z_0 = -1 \):

\[
\int_{\Gamma} \frac{\cos z}{(z + 1)(z + i)} \, dz = 2\pi i \cdot \frac{\cos(-1)}{-1 + i} = 2\pi i \cdot \frac{\cos(1)}{i - 1}
\]

Интеграл вокруг \( z = -i \)

Используем интегральную формулу Коши для функции \( f(z) = \frac{\cos z}{z + 1} \) с \( z_0 = -i \):

\[
\int_{\Gamma} \frac{\cos z}{(z + 1)(z + i)} \, dz = 2\pi i \cdot \frac{\cos(-i)}{-i + 1} = 2\pi i \cdot \frac{\cos(i)}{1 - i}
\]
Суммируем результаты
\[
\int_{\Gamma} \frac{\cos z}{(z + 1)(z + i)} \, dz = 2\pi i \left( \frac{\cos(1)}{i - 1} + \frac{\cos(i)}{1 - i} \right)
\]

b) \(\Gamma\) --- окружность \(|z - 1 - i| = 4\)

Оба простых нуля \( z = -1 \) и \( z = -i \) также находятся внутри этой окружности. Вычисления аналогичны случаю a:
\[
\int_{\Gamma} \frac{\cos z}{(z + 1)(z + i)} \, dz = 2\pi i \left( \frac{\cos(1)}{i - 1} + \frac{\cos(i)}{1 - i} \right)
\]

Значит как и в пункте а ответ:
\[
\int_{\Gamma} \frac{\cos z}{(z + 1)(z + i)} \, dz = 2\pi i \left( \frac{\cos(1)}{i - 1} + \frac{\cos(i)}{1 - i} \right)
\]
\section{Области сходимости рядов}
\subsection*{7.10}
Определить облась сходимости ряда
\[
\sum_{n=1}^{\infty}\frac{2^n+5}{(z-1+i)^n}
\]

\subsubsection*{Решение}
\[
\sum_{n=1}^{\infty} a_n (z - z_0)^n,
\]
где \( a_n = 2^n + 5 \) и \( z_0 = 1 - i \).


\[
R = \frac{1}{\limsup_{n \to \infty} \sqrt[n]{|a_n|}}.
\]

Найдём \(\limsup_{n \to \infty} \sqrt[n]{|a_n|}\):
\[
a_n = 2^n + 5.
\]

\[
\sqrt[n]{|a_n|} \approx \sqrt[n]{2^n} = 2.
\]

\[
\limsup_{n \to \infty} \sqrt[n]{|a_n|} = 2.
\]

\[
R = \frac{1}{2}.
\]

Таким образом, область сходимости ряда — круг с центром  \( 1 - i \) и радиусом \(\frac{1}{2}\):
\[
|z - (1 - i)| < \frac{1}{2}.
\]
\section{Разложение функций в ряды Тейлора и Лорана}
\subsection*{8.10}
Разложить функцию $f(z)=\frac{3z}{z^2-1}$ в ряд Тейлора в окрестности токи $z_0=0$. Найти радиус сходимости ряда.
\subsubsection*{Решение}

\[
f(z) = \frac{3z}{(z - 1)(z + 1)} = \frac{A}{z - 1} + \frac{B}{z + 1}
\]

\[
3z = A(z + 1) + B(z - 1)
\]

\[
3z = Az + A + Bz - B \implies 3z = (A + B)z + (A - B)
\]

\[
A + B = 3 \quad \text{и} \quad A - B = 0
\]

\[
A = B \quad \text{и} \quad 2A = 3 \implies A = \frac{3}{2}, \quad B = \frac{3}{2}
\]

\[
\frac{3z}{(z - 1)(z + 1)} = \frac{\frac{3}{2}}{z - 1} + \frac{\frac{3}{2}}{z + 1}
\]

Разложим каждую из этих дробей в ряд Тейлора в окрестности точки \( z = 0 \):
\[
\frac{3/2}{z - 1} = -\frac{3/2}{1 - z} = -\frac{3}{2} \sum_{n=0}^{\infty} z^n \quad \text{для} \ |z| < 1
\]
\[
\frac{3/2}{z + 1} = \frac{3/2}{1 + z} = \frac{3/2} \sum_{n=0}^{\infty} (-1)^n z^n \quad \text{для} \ |z| < 1
\]

\[
f(z) = -\frac{3}{2} \sum_{n=0}^{\infty} z^n + \frac{3}{2} \sum_{n=0}^{\infty} (-1)^n z^n
\]

\[
f(z) = \frac{3}{2} \left( \sum_{n=0}^{\infty} (-1)^n z^n - \sum_{n=0}^{\infty} z^n \right)
\]

Этот ряд Тейлора сходится в области \( |z| < 1 \). Таким образом, радиус сходимости \( R = 1 \).

\[
f(z) = \frac{3}{2} \left( \sum_{n=0}^{\infty} (-1)^n z^n - \sum_{n=0}^{\infty} z^n \right), \quad R = 1
\]
\subsection*{8.20}
Разложить функцию $f(z)=\frac{1-e^{-z}}{z^6}$ в ряд Лорана с центром в точке $z_0=0$. 
\subsubsection*{Решение}
\[
e^{-z} = \sum_{n=0}^{\infty} \frac{(-z)^n}{n!}
\]

\[
1 - e^{-z} = 1 - \sum_{n=0}^{\infty} \frac{(-z)^n}{n!} = 1 - \left( 1 - \frac{z}{1!} + \frac{z^2}{2!} - \frac{z^3}{3!} + \cdots \right)
\]
\[
1 - e^{-z} = z - \frac{z^2}{2!} + \frac{z^3}{3!} - \frac{z^4}{4!} + \frac{z^5}{5!} - \cdots
\]

\[
f(z) = \frac{1 - e^{-z}}{z^6} = \frac{z - \frac{z^2}{2!} + \frac{z^3}{3!} - \frac{z^4}{4!} + \frac{z^5}{5!} - \cdots}{z^6}
\]

\[
f(z) = \frac{z}{z^6} - \frac{\frac{z^2}{2!}}{z^6} + \frac{\frac{z^3}{3!}}{z^6} - \frac{\frac{z^4}{4!}}{z^6} + \frac{\frac{z^5}{5!}}{z^6} - \cdots
\]
\[
f(z) = z^{-5} - \frac{z^{-4}}{2!} + \frac{z^{-3}}{3!} - \frac{z^{-2}}{4!} + \frac{z^{-1}}{5!} - \cdots
\]

\[
f(z) = \sum_{n=1}^{\infty} \frac{(-1)^{n-1}}{n!} z^{-(6-n)}
\]
\subsection*{8.30}
Разложить функцию $f(z)=\frac{2z}{z^2-4z+3}$ в ряд Лорана в кольце $3<|z|<\infty$
\subsubsection*{Решение}
\[
z^2 - 4z + 3 = (z - 1)(z - 3)
\]

\[
f(z) = \frac{2z}{(z - 1)(z - 3)}
\]

\[
\frac{2z}{(z - 1)(z - 3)} = \frac{A}{z - 1} + \frac{B}{z - 3}
\]

\[
2z = A(z - 3) + B(z - 1)
\]
\[
2z = Az - 3A + Bz - B
\]
\[
2z = (A + B)z - 3A - B
\]

\[
A + B = 2
\]
\[
-3A - B = 0
\]

\[
B = 2 - A
\]
\[
-3A - (2 - A) = 0 \implies -3A - 2 + A = 0 \implies -2A - 2 = 0 \implies A = -1, \quad B = 3
\]

\[
\frac{2z}{(z - 1)(z - 3)} = \frac{-1}{z - 1} + \frac{3}{z - 3}
\]


Для \(\frac{1}{z - 1}\), так как \( |z| > 3 \):
\[
\frac{1}{z - 1} = \frac{1}{z\left(1 - \frac{1}{z}\right)} = \frac{1}{z} \cdot \frac{1}{1 - \frac{1}{z}} = \frac{1}{z} \sum_{n=0}^{\infty} \left(\frac{1}{z}\right)^n = \sum_{n=0}^{\infty} \frac{1}{z^{n+1}}
\]

Для \(\frac{1}{z - 3}\), так как \( |z| > 3 \):
\[
\frac{1}{z - 3} = \frac{1}{z\left(1 - \frac{3}{z}\right)} = \frac{1}{z} \cdot \frac{1}{1 - \frac{3}{z}} = \frac{1}{z} \sum_{n=0}^{\infty} \left(\frac{3}{z}\right)^n = \sum_{n=0}^{\infty} \frac{3^n}{z^{n+1}}
\]

\[
f(z) = -\sum_{n=0}^{\infty} \frac{1}{z^{n+1}} + 3 \sum_{n=0}^{\infty} \frac{3^n}{z^{n+1}}
\]

\[
f(z) = \sum_{n=0}^{\infty} \left(3^{n+1} - 1\right) \frac{1}{z^{n+1}}
\]

\[
f(z) = \sum_{n=0}^{\infty} \left(3^{n+1} - 1\right) \frac{1}{z^{n+1}}
\]
\section{Изолированные особые точки. Вычеты}
\subsection*{9.10}
Найти все конечные изолированные особые точки функции $f(z)=z\cos{\frac{1}{z^2}}$ и определить характер(нордический) каждой из них
\subsubsection*{Решение}
Характер особой точки в \( z = 0 \)
\[
\cos w = \sum_{n=0}^{\infty} \frac{(-1)^n w^{2n}}{(2n)!}
\]
\[
\cos \frac{1}{z^2} = \sum_{n=0}^{\infty} \frac{(-1)^n}{(2n)! z^{4n}}
\]

\[
f(z) = z \cos \frac{1}{z^2} = z \sum_{n=0}^{\infty} \frac{(-1)^n}{(2n)! z^{4n}} = \sum_{n=0}^{\infty} \frac{(-1)^n}{(2n)!} z^{1 - 4n}
\]
\[
f(z) = \sum_{n=0}^{\infty} \frac{(-1)^n}{(2n)!} z^{1 - 4n}
\]
Из этого разложения видно, что \( z = 0 \) является существенной особой точкой, так как ряд содержит бесконечное число членов с отрицательными степенями \( z \).

\subsection*{9.20}
Указать характер изолированной особой точки $z_0=\infty$ функции $f(z)=z\cos{\frac{1}{2z^3+z}}$
\subsubsection*{Решение}

\[
\cos x = 1 - \frac{x^2}{2!} + \frac{x^4}{4!} - \cdots
\]
Посмотрев что в косинусе стоит $\frac{1}{2z^3+z}}$ становится очевидно, что будет бесконечно много членов отрицательной степени, значит $z_0=\infty$ существенная особая точка

\subsection*{9.30}
Найти вычеты функции $f(z)=\frac{\cosh{z}}{(z-i)(z-1)^2}$ в конечных изолированных особых точках
\subsubsection*{Решение}
\[
\frac {e^z+e^{-z}}{2}=0 \iff e^{2z}=-1
\]
\[
2z=i\pi(2k+1), k \in \mathbb{Z}
\]
\[
z=0+\left(\frac{\pi}2+k\pi\right) i, k \in \mathbb{Z}
\]
В точке \( z = i \) функция имеет простой полюс, а В точке \( z = 1 \)  полюс второго порядка.

Вычет в точке \( z = i \)
\[
\operatorname{Res}(f, i) = \lim_{z \to i} (z - i) f(z)
\]

\[
\operatorname{Res}(f, i) = \lim_{z \to i} (z - i) \frac{\cosh z}{(z - i)(z - 1)^2} = \lim_{z \to i} \frac{\cosh z}{(z - 1)^2} = \frac{\cosh i}{(i - 1)^2}
\]

Вычет в точке \( z = 1 \)

\[
\operatorname{Res}(f, 1) = \lim_{z \to 1} \frac{d}{dz} \left[ (z - 1)^2 f(z) \right]
\]

\[
(z - 1)^2 f(z) = \frac{\cosh z}{z - i}
\]

\[
\frac{d}{dz} \left[ \frac{\cosh z}{z - i} \right] = \frac{(z - i) \sinh z - \cosh z}{(z - i)^2}
\]

\[
\operatorname{Res}(f, 1) = \frac{(1 - i) \sinh 1 - \cosh 1}{(1 - i)^2}
\]
\subsection*{9.40}
Найти вычет функции $f(z)=z^2\cos{\frac{1}{z}}$ в точке $z_0=\infty$
\subsubsection*{Решение}

\[
z^2\cos{\frac{1}{z}} = z^2(1 - \frac{z^{-2}}{2!} + \frac{x^{-4}}{4!} - \cdots)
\]
Очевидно, что в разложении нашей функции будет присутствовать бесконечно много отрицательных степеней, значит вычет равен нулю
\[
 \operatorname{Res}(f, \infty) = 0 
\]
\section{Вычисление с помощью вычетов комплексныъ интегралов}
\subsection*{10.10}
Вычислить интеграл
\[
\int_{|z - 1 + i| = 2}^{} \frac{1}{(z - 1 + i)(z - 2 + i)} \, dz
\]
с помощью теоремы о вычетах.
\subsubsection*{Решение}
\[ 
(z = 1 - i): |1 - i - (1 - i)| = 0 < 2
\]
\[
 (z = 2 - i): |2 - i - (1 - i)| = |1| = 1 < 2
\]

Оба полюса находятся внутри контура.

Для полюса в \(z = 1 - i\):
\[
\operatorname{Res}\left( \frac{1}{(z - 1 + i)(z - 2 + i)}, z = 1 - i \right) = \frac{1}{(1 - i - 2 + i)} = \frac{1}{-1} = -1
\]

Для полюса в \(z = 2 - i\):
\[
\operatorname{Res}\left( \frac{1}{(z - 1 + i)(z - 2 + i)}, z = 2 - i \right) = \frac{1}{(2 - i - 1 + i)} = \frac{1}{1} = 1
\]

По теореме о вычетах, интеграл по замкнутому контуру равен \(2\pi i\) умноженному на сумму вычетов функции внутри контура:
\[
\int_{|z - 1 + i| = 2} \frac{1}{(z - 1 + i)(z - 2 + i)} \, dz = 2\pi i \left( -1 + 1 \right) = 2\pi i \cdot 0 = 0
\]
\subsection*{10.20}
Вычислить интеграл
\[
\int_{|z - 1 - i| = 10} \frac{e^z}{z^2(z + i)} \, dz
\]
с помощью теоремы о вычетах.
\subsubsection*{Решение}
Функция \( \frac{e^z}{z^2(z + i)} \) имеет полюсы в точках \( z = 0 \) (полюс второго порядка) и \( z = -i \) (простой полюс), которые находятся внутри контура

Для полюса в \(z = 0\) (полюс второго порядка):
\[
\operatorname{Res}\left( \frac{e^z}{z^2(z + i)}, z = 0 \right) = \lim_{z \to 0} \frac{d}{dz} \left[ z^2 \frac{e^z}{z^2(z + i)} \right] = \lim_{z \to 0} \frac{d}{dz} \left[ \frac{e^z}{z + i} \right]
\]
\[
\frac{d}{dz} \left[ \frac{e^z}{z + i} \right] = \frac{(z + i) e^z - e^z}{(z + i)^2} = \frac{z e^z}{(z + i)^2}
\]

\[
\operatorname{Res}\left( \frac{e^z}{z^2(z + i)}, z = 0 \right) = \frac{0 \cdot e^0}{(0 + i)^2} = 0
\]

Для полюса в \(z = -i\) (простой полюс):
\[
\operatorname{Res}\left( \frac{e^z}{z^2(z + i)}, z = -i \right) = \frac{e^{-i}}{-1} = -e^{-i}
\]

По теореме о вычетах, интеграл по замкнутому контуру равен \(2\pi i\) умноженному на сумму вычетов функции внутри контура:
\[
\int_{|z - 1 - i| = 10} \frac{e^z}{z^2(z + i)} \, dz = 2\pi i \left( 0 - e^{-i} \right) = -2\pi i e^{-i}
\]
\section{Вычисление с помощью вычетов комплексныъ интегралов}
\subsection*{11.10}
Вычислить интеграл
\[
\int_{-\infty}^{\infty} \frac{1}{x^2 - 8x + 25} \, dx
\]
методами теории функций комплексного переменного
\subsubsection*{Решение}
Мы можем использовать, тк при $\epsilon=1.1,M=1$ выполняется (11.1)
\[
\int_{-\infty}^{\infty} f(x) \, dx = 2\pi i \sum_{k=1}^n \operatorname{Res} \left[ f(z); z_k \right].
\]

Преобразуем подынтегральное выражение:
\[
x^2 - 8x + 25 = (x - 4)^2 + 9 = (x - 4)^2 + 3^2.
\]

\[
f(z) = \frac{1}{(z - 4)^2 + 9}.
\]
Эта функция имеет полюсы в точках:
\[
z = 4 \pm 3i.
\]
\[
\operatorname{Res}\left[ \frac{1}{(z - 4)^2 + 9}, z = 4 + 3i \right] = \lim_{z \to 4 + 3i} (z - (4 + 3i)) \frac{1}{(z - (4 + 3i))(z - (4 - 3i))}=\]
\[
= \frac{1}{(4 + 3i) - (4 - 3i)} = \frac{1}{6i} = -\frac{i}{6}.
\]

\[
\int_{-\infty}^{\infty} \frac{1}{x^2 - 8x + 25} \, dx = \frac{\pi}{3}.
\]
\end{document}